\section{Quantum Information}


\begin{frame}
\begin{refsection}
	
\vfill

\vspace{-.5cm}\textbf{\Large A Knowledge Compilation Map for Quantum Information}~\cite{vinkhuijzen2024a}\vspace{-.5cm}

\vfill

\printbibliography[section=\therefsection]
\end{refsection}

\end{frame}



\begin{refframe}{Knowledge Representation}


    \begin{block}{A knowledge compilation map~\cite{darwiche2002knowledge}}
    \begin{itemize}
    \item  Normal forms for representing Boolean functions: $f(v_1,\dots, v_n) \rightarrow \mathbb B$
    \item Tradeoff: succinctness vs tractability
    \end{itemize}  
    \end{block}

\pause

\includegraphics[width=4cm]{concise}
\includegraphics[width=6cm]{ops}

\end{refframe}



\begin{frame}{Comparing Decision Diagrams vs RBM vs MPS}

\vspace{-.5em}
\centering

%\begin{figure}[b] 
~\begin{tikzpicture}[-{>[scale=0.3]},>=stealth',shorten >=1pt,auto,node distance=.4cm,
    thick, state/.style={circle,draw,minimum size=10pt,inner sep=.6pt},font=\scriptsize]

\node (vec) {
    \begin{minipage}{1.3cm}\footnotesize
$\def\arraystretch{1.}
    \begin{bmatrix*}[c]
    \frac1{\sqrt 2} \\ 0 \\ 0 \\ 0 \\ 0 \\ 0 \\ 0 \\ \frac1{\sqrt 2}
    \end{bmatrix*}
    $
    \end{minipage}
};


\node[state, right = 2cm of vec.north,anchor=north,yshift=-.2cm] (n1) {$x_3$};

\node[state](n2)[below = of n1, xshift=-.9cm]{$x_2$};
\node[state](n3)[below = of n1, xshift= .9cm]{$x_2$};

\node[state](n21)[below = of n2, ]{$x_1$};
\node[state](n22)[below = of n2, xshift= .9cm]{$x_1$};

\node[state](n31)[below = of n3]{$x_1$};

\node[draw, rectangle,minimum size=.5cm,below = of n21, xshift=.2cm,minimum size=10pt,inner sep=1pt] (l1) {$0$};
\node[draw, rectangle,minimum size=.5cm,right = of l1,minimum size=10pt,inner sep=1pt] (l2) {$\nicefrac1{\sqrt 2}$};


\path[]
(n1) edge[e1] node[right,pos=.7] {} (n3)
(n1) edge[e0] node[left,pos=.7] {} (n2)
(n2) edge[e0] node[left,pos=.7] {} (n21)
(n2) edge[e1] node[left,pos=.7] {} (n22)
(n3) edge[e0] node[left,pos=.7] {} (n22)
(n3) edge[e1] node[left,pos=.7] {} (n31)
(n21) edge[e0=  0]  node[pos=.7] {} (l2)
(n21) edge[e1=  0]  node[pos=.7] {} (l1)
(n22) edge[e0= 20]  node[pos=.7] {} (l1)
(n22) edge[e1= 20]  node[pos=.7] {} (l1)
(n31) edge[e0=  0]  node[pos=.7] {} (l1)
(n31) edge[e1=  0]  node[pos=.7] {} (l2)
;

\node[above=.5cm of n1,anchor=north]    (add) {\textbf{\add}};
\node[left =1.3cm of add]    (svc) {\textbf{Vector}};


\node[state, right=2cm of n1] (q1) {$x_3$} ;
\node[state, below=of q1,xshift=-0.5cm] (q2) {$x_2$} ;
\node[state, below=of q1,xshift=0.5cm] (q3) {$x_2$} ;
\node[state, below=of q2] (q4) {$x_1$} ;
\node[state, below=of q3] (q5) {$x_1$} ;
\node[draw, rectangle,minimum size=.5cm,below = of q4,xshift=0.5cm,minimum size=10pt,inner sep=1pt] (l1) {$1$};


\draw[e0] (q1) edge  node[] {} (q2);
\draw[e1] (q1) edge  node[] {} (q3);
\draw[e0=25] (q2) edge  node[] {} (q4);
\draw[e1=25] (q2) edge  node[,pos=.45] {$0$} (q4);
\draw[e0=25] (q3) edge  node[] {} (q5);
\draw[e1=25] (q3) edge  node[,xshift=-0.6cm,pos=.45] {$0$} (q5);
\draw[e0=-20] (q4) edge  node[] {} (l1);
\draw[e1=-20] (q4) edge  node[,left=.1cm,pos=.45] {$0$} (l1);
\draw[e0=-20] (q5) edge  node[] {} (l1);
\draw[e1=-20] (q5) edge  node[,xshift=0.4cm,pos=.45] {$0$} (l1);
\draw[<-] (q1) --++(90:.5cm) node[right=.2cm,pos=.7,] {$\nicefrac 1{\sqrt 2}$} node[right,pos=.8] {};

\node[right =1.7cm of add]    (sldd) {\textbf{QMDD}};



    \node[state, right = 1.8cm of q1] (a1) {$x_3$};
    \node[state, below = of a1] (a3) {$x_2$};
    \node[state, below = of a3] (a4) {$x_1$};
    \node[draw,rectangle,minimum size=0.5cm, below= of a4,minimum size=10pt,inner sep=1pt] (w4) {1};


    \draw[<-] (a1) --++(90:.5cm) node[right=.2cm,pos=.7,] {$\nicefrac 1{\sqrt 2} \cdot \id^{\otimes3}$} node[left,pos=.8] {};
    \draw[e0=25] (a1) edge  node[] {} (a3);
    \draw[e1=25] (a1) edge  node[pos=.3,right] {$X \otimes X$} (a3);
    \draw[e0=25] (a1) edge  node[] {} (a3);
    \draw[e1=25] (a3) edge  node[pos=.3,right] {$0$} (a4);
    \draw[e0=25] (a3) edge  node[] {} (a4);
    \draw[e0=25] (a4) edge  node[] {} (w4);
    \draw[e1=25] (a4) edge  node[pos=.3,right] {0} (w4);

\node[right=1.4cm of sldd]    (limdd) {\textbf{\limdd}};


%\pause

\node[state, xshift=-5cm, below=.7cm of l1,yshift=.cm] (h1) {$h_1$} ;
\node[state, right=.65cm of h1] (h2) {$h_2$} ;
\node[state, right=.65cm of h2] (h3) {$h_3$} ;
\node[state, right=.65cm of h3] (h4) {$h_4$} ;
\node[state, below=1.7cm of h1, xshift=0.5cm] (v1) {$v_1$} ;
\node[state, right=.7cm of v1] (v2) {$v_2$} ;
\node[state, right=.7cm of v2] (v3) {$v_3$} ;


\draw[e1] (h1) edge  node[] {} (v1);
\draw[e1=-10] (h1) edge  node[] {} (v2);
\draw[e1=-10] (h1) edge  node[] {} (v3);
\draw[e1=10] (h2) edge  node[] {} (v1);
\draw[e1] (h2) edge  node[] {} (v2);
\draw[e1=-10] (h2) edge  node[] {} (v3);
\draw[e1=10] (h3) edge  node[] {} (v1);
\draw[e1] (h3) edge  node[] {} (v2);
\draw[e1=-10] (h3) edge  node[] {} (v3);
\draw[e1=10] (h4) edge  node[] {} (v1);
\draw[e1=10] (h4) edge  node[] {} (v2);
\draw[e1] (h4) edge  node[] {} (v3);


\node[draw,rectangle, below=0.2cm of h1, fill=white, opacity=1] (1) {$i\pi/3$};
\node[draw,rectangle, below=0.2cm of h2, fill=white, opacity=1] (1) {$i\pi/3$};
\node[draw,rectangle, below=0.2cm of h3, xshift=-0.2cm, fill=white, opacity=1] (1) {$-i\pi/3$};
\node[draw,rectangle, below=0.2cm of h4, fill=white, opacity=1] (1) {$-i\pi/3$};


\node[above=0.cm of h3,xshift=-0.6cm]    (a31) {{hidden layer}};
\node[below=0.cm of v2,xshift=0.1cm]    (a31) {{visible layer}};

\node[below=.2cm of l1,xshift=-3.3cm]    (rbm) {\textbf{RBM}};



\node[right=.7cm of h4] (a30) {$A_3^0=\begin{bmatrix}1 & 0\end{bmatrix}$};
\node[right=2cm of a30.west, anchor=west]   (a31) {$A_3^1=\begin{bmatrix}0 & 1\end{bmatrix}$};
\node[below=1.cm of a30.west, anchor=west] (a20) {$A_2^0=\begin{bmatrix}1 & 0 \\ 0 & 0\end{bmatrix}$};
\node[below=1.cm of a31.west, anchor=west] (a21) {$A_2^1=\begin{bmatrix}0 & 0 \\ 0 & 1\end{bmatrix}$};
\node[below=1.cm of a20.west, anchor=west] (a10) {$A_1^0=\begin{bmatrix}\frac1{\sqrt 2} \\ 0\end{bmatrix}$};
\node[below=1.cm of a21.west, anchor=west] (a11) {$A_1^1=\begin{bmatrix}0 \\ \frac1{\sqrt 2}\end{bmatrix}$};

\node[right=3.8cm of rbm]    (a31) {\textbf{MPS}};



\end{tikzpicture}

%\caption{The $3$-qubit GHZ state $\nicefrac{1}{\sqrt 2}(\ket{000} + \ket{111})$, displayed using different data structures.
%The unlabelled edges for \add, \qmdd, \limdd have resp. label 1, 1, $\id$.
%In the RBM, the weights of edges incident to $h_1,h_2$ ($h_3, h_4$) are all $i\pi/3$ ($-i\pi/3$); the hidden node biases $(\beta_{h_1}, \beta_{h_2}, \beta_{h_3}, \beta_{h_4}) = i\pi \cdot (1/3, 2/3, -1/3, -2/3)$; the visible node biases $\alpha_{v_1}=\alpha_{v_2}=\alpha_{v_3}=0$.
%}
%\label{fig:ghz-examples}
%\end{figure}

\pause

\begin{block}{Compare Different Representations Analytically}
	\begin{itemize}\vspace{-.5em}
		\item \textbf{Succinctness}: Families of quantum states showing exponential separations
		\item \textbf{Tractability}: Is manipulation of the representation (e.g., applying a gate) efficient
	\end{itemize}\vspace{-.5em}
\end{block}

\end{frame}




\begin{frame}{Matrix Product States}

\begin{definition}[MPS on $n$ qubits]
	An MPS $M$ is a series of $2n$ matrices of the right dimensions.
	\vspace{.5em}
	
	$A_1^0,~~~ A_2^0,~~~ \dots,~~~ A_n^0$\\
	\vspace{.5em}
	$A_1^1,~~~ A_2^1,~~~ \dots,~~~ A_n^1$
	
	\vspace{.5em}

The interpretation $\ket{M}$ is determined as $\braket{\vec x | M} = A_n^{x_n} \cdot  \cdots  \cdot A_2^{x_2}  \cdot A_1^{x_1}$ for $\vec x \in \{0, 1\}^n$.
\end{definition}

\pause

\begin{exampleblock}{MPS}
	\begin{tikzpicture}[-{>[scale=0.3]},>=stealth',shorten >=1pt,auto,node distance=.4cm,
    thick, state/.style={circle,draw,minimum size=10pt,inner sep=.6pt},font=\scriptsize]

\node (vec) {
    \begin{minipage}{1.3cm}\footnotesize
$\def\arraystretch{1.}
    \begin{bmatrix*}[c]
    \frac1{\sqrt 2} \\ 0 \\ 0 \\ 0 \\ 0 \\ 0 \\ 0 \\ \frac1{\sqrt 2}
    \end{bmatrix*}
    $
    \end{minipage}
};


\node[right=.7cm of vec, yshift=1cm] (a30) {$A_3^0=\begin{bmatrix}1 & 0\end{bmatrix}$};
\node[right=2cm of a30.west, anchor=west]   (a31) {$A_3^1=\begin{bmatrix}0 & 1\end{bmatrix}$};
\node[below=1.cm of a30.west, anchor=west] (a20) {$A_2^0=\begin{bmatrix}1 & 0 \\ 0 & 0\end{bmatrix}$};
\node[below=1.cm of a31.west, anchor=west] (a21) {$A_2^1=\begin{bmatrix}0 & 0 \\ 0 & 1\end{bmatrix}$};
\node[below=1.cm of a20.west, anchor=west] (a10) {$A_1^0=\begin{bmatrix}\frac1{\sqrt 2} \\ 0\end{bmatrix}$};
\node[below=1.cm of a21.west, anchor=west] (a11) {$A_1^1=\begin{bmatrix}0 \\ \frac1{\sqrt 2}\end{bmatrix}$};


\end{tikzpicture}

\end{exampleblock}

	
\end{frame}


\begin{frame}{QMDD vs MPS}


\begin{theorem}
MPS is at least as succinct as QMDD.	
\end{theorem}


\pause

\textbf{Proof:} 

\centering
	\begin{tikzpicture}[-{>[scale=0.3]},>=stealth',shorten >=1pt,auto,node distance=.4cm,
    thick, state/.style={circle,draw,minimum size=10pt,inner sep=.6pt},font=\scriptsize]


\node[yshift=1cm] (a30) {$A_3^0=\begin{bmatrix}1 & 0\end{bmatrix}$};
\onslide<+(1)->{
\node[right=2cm of a30.west, anchor=west]   (a31) {$A_3^1=\begin{bmatrix}0 & 1\end{bmatrix}$};
}
\node[below=1.cm of a30.west, anchor=west] (a20) {$A_2^0=\begin{bmatrix}1 & 0 \\ 0 & 0\end{bmatrix}$};
\onslide<.(1)->{
\node[below=1.cm of a31.west, anchor=west] (a21) {$A_2^1=\begin{bmatrix}0 & 0 \\ 0 & 1\end{bmatrix}$};
}
\node[below=1.cm of a20.west, anchor=west] (a10) {$A_1^0=\begin{bmatrix}\frac1{\sqrt 2} \\ 0\end{bmatrix}$};
\onslide<.(1)->{
\node[below=1.cm of a21.west, anchor=west] (a11) {$A_1^1=\begin{bmatrix}0 \\ \frac1{\sqrt 2}\end{bmatrix}$};
}


\node[state, right=2cm of a31] (q1) {$x_3$} ;
\node[state, below=of q1,xshift=-0.5cm] (q2) {$x_2$} ;
\node[state, below=of q1,xshift=0.5cm] (q3) {$x_2$} ;
\node[state, below=of q2] (q4) {$x_1$} ;
\node[state, below=of q3] (q5) {$x_1$} ;
\node[draw, rectangle,minimum size=.5cm,below = of q4,xshift=0.5cm,minimum size=10pt,inner sep=1pt] (l1) {$1$};


\draw[e0] (q1) edge  node[] {} (q2);
\onslide<.(1)->{
\draw[e1] (q1) edge  node[] {} (q3);
}
\draw[e0=25] (q2) edge  node[] {} (q4);
\onslide<.(1)->{
\draw[e1=25] (q2) edge  node[,pos=.45] {$0$} (q4);
}
\draw[e0=25] (q3) edge  node[] {} (q5);
\onslide<.(1)->{
\draw[e1=25] (q3) edge  node[,xshift=-0.6cm,pos=.45] {$0$} (q5);
}
\draw[e0=-30] (q4) edge  node[above ] {$\frac 1{\sqrt 2}$} (l1);
\onslide<.(1)->{
\draw[e1=-20] (q4) edge  node[,left=.1cm,pos=.45] {$0$} (l1);
}
\draw[e0=-40] (q5) edge  node[right] {$0$} (l1);
\onslide<.(1)->{
\draw[e1=-20] (q5) edge  node[above,xshift=0.cm,pos=.45] {$\frac 1{\sqrt 2}$} (l1);
}
\draw[<-] (q1) --++(90:.5cm) node[right=.2cm,pos=.7,] {} node[right,pos=.8] {};

%\node[right =1.7cm of add]    (sldd) {\textbf{QMDD}};

\end{tikzpicture}


\pause

\begin{theorem}
MPS is exponentially more succinct than QMDD.	
\end{theorem}

\[\ket \phi = \sum_{x \in \set{0,1}^n} (x)_2 \ket{x} \]

	
\end{frame}



\begin{frame}{\limdd vs MPS}


\begin{lemma}
	There is a family of quantum states with polynomial-size \limdd but exponential-size MPS.
\end{lemma}

\vspace{-.5em}

\textbf{Proof:} It is well-known that MPS can be exponentially sized for stabilizer states.


\vspace{.5em}

\pause


\begin{lemma}
	There is a family of quantum states with polynomial-size MPS but exponential-size \limdd.
\end{lemma}


\vspace{-.5em}
\textbf{Proof sketch:} 

Take the state: \vspace{-2.5em}
\begin{align}
\sumstate = \ket{+}^{\otimes n} + \bigotimes_{j=1}^n(\ket 0+e^{i\pi 2^{-j-1}}\ket 1)
\end{align}


\pause
MPS: \vspace{-2.5em}
\begin{align*}
	    A^{0}_1 = \begin{bmatrix} 1 & 1\end{bmatrix}, ~~~~  &A^{x_j}_j = \begin{bmatrix} 1 &  0 \\ 0 & 1 \end{bmatrix}, ~~~~~~ &&A^{x_n}_n = \begin{bmatrix} 1 \\ 1\end{bmatrix}&\\
	    A^{x_1}_1 = \begin{bmatrix} 1 & e^{i\pi 2^{-2}}\end{bmatrix}, ~~~~  &A^{x_j}_j = \begin{bmatrix} 1 &  0 \\ 0 & \prod_{j=2}^{n-1}e^{i\pi 2^{-j-1}}\end{bmatrix},  && A^{x_n}_n = \begin{bmatrix} 1 \\ e^{i\pi 2^{-n-1}}&\end{bmatrix},
	\end{align*}	

\pause
LIMDD:

There is no LIM mapping subfunctions $f_{\vec a}$ 
to each other for different ${\vec a}\in \set{0,1}^{n-1}$:
\[
f_{\vec a}(x_1) = 1 + e^{i\pi \sum_{j=2}^n a_j \cdot 2^{-j-1}} \cdot
e^{i\pi \cdot \nicefrac14 x_1 } \ket{x_1}
\]

\end{frame}



\begin{frame}{\limdd vs RBM}
\begin{refsection}


\begin{lemma}
	There is a family of quantum states with polynomial-size RBM but exponential-size \limdd.
\end{lemma}

%\textbf{Proof:}
%
%\begin{columns}[T]
%\begin{column}{0.6\textwidth}
%We use the seminal Boolean function $IP:  \vec{x}, \vec{y} \mapsto  \vec{x}^T  \vec{y} \mod 2$ for even $n$, which computes the inner product between the first half of the input with the second half.
%
%~\\
%
%\citeauthor{martens2013representational}~\cite{martens2013representational} shows RBM is exponential for it.
%\end{column}
%\begin{column}{0.3\textwidth}
%\vspace{-3em}
%\scalebox{.5}{
%\begin{tikzpicture}[
%    scale=0.3,
%    every path/.style={>=latex},
%    every node/.style={},
%    inner sep=0pt,
%    minimum size=14pt,
%    line width=1pt,
%    node distance=.3cm,
%    thick,
%    font=\footnotesize
%    ]
%
%    \node[] (root)   {};
%    \node[draw,circle, below =of root, xshift=0cm] (l1) {};
%    \node[draw,circle, below =of l1] (l2) {};
%
%    \node[draw,circle, below =of l1, xshift=1cm] (ma2) {};
%
%    \node[draw,circle, below =of l2] (l3) {};
%    \node[draw,circle, below =of l2, xshift=3cm] (r3) {};
%    \node[draw,circle, below =of l3] (l4) {};
%    \node[draw,circle, below =of r3] (r4) {};
%    \node[draw,circle, below =of l3, xshift=1cm] (ma4) {};
%    \node[draw,circle, below =of l3, xshift=2cm] (mb4) {};
%
%    \node[draw,circle, below =.6cm of l4] (l5) {};
%    \node[draw,circle, below =.6cm of r4] (r5) {};
%
%    \node[draw,circle, below= of l5] (l6) {};
%    \node[draw,circle, below= of r5] (r6) {};
%    \node[draw,circle, below= of l5, xshift=1cm] (ma6) {};
%    \node[draw,circle, below= of l5, xshift=2cm] (mb6) {};
%
%	 \node[left of =l1] 	{$x_1$};
%	 \node[left of =l2] 	{$x_2$};
%	 \node[left of =l3] 	{$x_3$};
%	 \node[left of =l4] 	{$x_4$};
%	 \node[left  =.1cm of l5] 	{$x_{n-1}$};
%	 \node[left of =l6] 	{$x_{n}$};
%
%    \node[draw,circle,rectangle,minimum size=0.4cm, below=of l6] (leaf0) {$0$};
%    \node[draw,circle,rectangle,minimum size=0.4cm, below=of r6] (leaf1) {$\frac{1}{A}$};
%
%    \draw[<-] (l1) --++(90:2cm) node[pos=1.4,right] {};
%
%    \draw[e0=0] (l1) edge  node[] {} (l2);
%    \draw[e1=0] (l1) edge  node[] {} (ma2);
%
%
%    \draw[e0=25] (l2) edge  node {} (l3);
%    \draw[e1=25] (l2) edge  node {} (l3);
%
%
%    \draw[e0=0] (ma2) edge  node {} (l3);
%    \draw[e1=0] (ma2) edge  node {} (r3);
%
%
%    \draw[e0=0] (l3) edge  node {} (l4);
%    \draw[e1=0] (l3) edge  node {} (ma4);
%
%    \draw[e1=0] (r3) edge  node {} (mb4);
%    \draw[e0=0] (r3) edge  node[left] {} (r4);
%
%    \node[ xshift=0.5cm, yshift=.5cm, below= of ma4] (a) {$\pmb{\vdots}$};
%
%    \draw[e0=0] (l5) edge  node {} (l6);
%    \draw[e1=0] (l5) edge  node {} (ma6);
%
%    \draw[e0=0] (r5) edge  node {} (r6);
%    \draw[e1=0] (r5) edge  node {} (mb6);
%
%    \draw[e0=0] (ma6) edge  node {} (leaf0);
%    \draw[e1=0] (ma6) edge  node {} (leaf1);
%
%    \draw[e0=0] (mb6) edge  node {} (leaf1);
%    \draw[e1=0] (mb6) edge  node {} (leaf0);
%
%    \draw[e0=25] (l6) edge  node {} (leaf0);
%    \draw[e1=25] (l6) edge  node {} (leaf0);
%
%    \draw[e1=25] (r6) edge  node {} (leaf1);
%    \draw[e0=25] (r6) edge  node {} (leaf1);
%\end{tikzpicture}
%}
%\end{column}
%\end{columns}

	
%\onslide<+->{


\begin{lemma}
	There is a family of quantum states with polynomial-size \limdd but exponential-size RBM.
\end{lemma}

%We use the $\ket{Sum}$ state again.
%}

%\printbibliography[section=\therefsection]
\end{refsection}
\end{frame}



\begin{refsection}
\begin{frame}{A Knowledge Compilation Map for Quantum Information}

\begin{theorem}[Decision Diagram vs MPS vs RBM~\cite{vinkhuijzen2024a}]
\begin{itemize}
%	\item is a generalization of QMDD
%	\item Nodes are merged up to $\gamma \cdot P_1 \otimes \dots \otimes P_n$ with $P_1, \dots, P_n \in \set{\id, X, Y, Z}$
%	\item<2-> All stabilizer states have linear \limdd size
%	\item<3-> Stabilizer (Clifford) circuit simulation is polynomial time in \limdd
%\pause
%\pause
%	\item RBM is quantum neural network
%	\item MPS is a tensor train (linear tensor network)
	\item<+-> \limdd \& MPS are exponentially more succinct than QMDD \& ADD
	\item<+-> \limdd, MPS \& RBM are incomparable
	\item<+-> Some operations become harder on \limdd
\end{itemize}

\centering

\begin{columns}%[onlywidth,T]
	\begin{column}{.35\textwidth}
\phantom{ZZZZ}
\scalebox{.63}{
\hspace{2em}
\begin{tikzpicture}[node distance=.7cm,minimum height=.5cm]
        % nodes
		\node[draw] (mps) 			   {MPS};
		\node[draw, right =2cm of mps, yshift=-2.1cm] (limdd) {\limdd};
%		\node[draw, above =of limdd, yshift=1.5cm] (tn) {TN};
\onslide<2->{
		\node[draw, right =2cm of limdd, yshift=2.1cm] (rbm) {RBM};
}
%		\node[draw, below =2cm of rbm, xshift=0.4cm,text width=1.3cm, align=center] (ss) {Stabilizer States};
		\node[draw, below = 1cm of limdd] (qmdd) {QMDD};
		\node[draw, below = 1cm of qmdd] (add) {ADD};
%		\node[draw, below = 1cm of add, xshift= 0cm] (statevec) {Vector};

        % edges MPS
\onslide<2->{

		\draw[\bigarrowheadb, bend left=0] (limdd.north west) to[<->] node[midway] {$\boldsymbol{\bigtimes}$} 
											node[pos=.8,right] {\supp{\autoref{thm:succ-limdd-vs-mps}}}
											node[pos=.2,right] {\supp{\autoref{thm:succ-mps-vs-limdd}}}
											(mps.south east);
}

		\draw[\bigarrowhead, bend left=-10] (mps) to (qmdd.west);

		\draw[\bigarrowhead, bend left=-20] (mps.south west) to
													 (add.west);

\onslide<2->{

        % edges RBM
		\draw[\bigarrowheadb, bend left=0] (rbm.west) to node[midway] {$\boldsymbol{\bigtimes}$} 
											(mps.east);

		\draw[\bigarrowheadb, bend right=0] (rbm.south west) to node[midway] {$\boldsymbol{\bigtimes}$} 
											node[pos=.7,above left=-1mm] {\supp{\autoref{thm:succ-rbm-vs-limdd}}} 
											(limdd.north east);

		\draw[\bigarrowheadb, bend left=10] (rbm) to node[midway] {$\boldsymbol{\bigtimes}$} 
										   (qmdd.east);

		\draw[\bigarrowheadb, bend left=20] (rbm.south east) to node[midway] {$\boldsymbol{\bigtimes}$}
											node[pos=.3,below right=-1mm] {\supp{\autoref{thm:add-rbm}}} 
											 (add.east);
}


		%edges QDD
		\draw[\bigarrowhead, bend right=0] (limdd) to (qmdd);
		\draw[\bigarrowhead, bend right=0] (qmdd) to  (add);

	\end{tikzpicture}
}
\footnotesize
\\
\hspace{3em}
~~~~~~Succinctness separations.
%\phantom{ZZ}
%\phantom{ZZ}\begin{tikzpicture}\footnotesize
%  \tikzset{venn circle/.style={circle,minimum width=2cm,fill=####1,opacity=0.4}}
%  \node [venn circle=white,minimum width=4cm,draw] (A) at (0,0.3) {};
%  \node  at (0,1.95) 			{State space};
%
%  \node [venn circle = Red!40!white, ellipse,minimum height=2.2cm, minimum width=3.6cm] (L) at (0,0.6) {};
%  \node  at (0,1.5) 		{\limdd};
%  
%  \node [venn circle = blue!70!white,text width=1.3cm,align=center,rotate=79,ellipse,minimum height=1.8cm, minimum width=2.5cm] (B) at (-.6,-.2) {};
%  \node[text width=1.3cm,align=center]  at (-.9,-1.) {MPS};
%
%
%  \node [venn circle = green!70!white,text width=1.3cm,align=center,rotate=119,ellipse,minimum height=1.8cm, minimum width=2.5cm] (B) at (.6,-.2) {};
%  \node[text width=1.3cm,align=center]  at (.9,-1.) {RBM};
%
%
%
%  \node [venn circle = Blue!100!white,text width=1cm,align=center, minimum width=1.cm] (B) at (-.5,.3) 	{\textcolor{white}{QMDD}};						
%
%  \node [venn circle = OliveGreen!100!white,text width=1cm,align=center, minimum width=1.cm] (C) at (.5,.2) {\textcolor{white}{Stabilizer states}};
%
%\end{tikzpicture}
	\end{column}
	\begin{column}{.65\textwidth}
\onslide<3->{
\setlength{\tabcolsep}{2pt}
\def\arraystretch{1.1}
\footnotesize
\centering
\begin{tabular}{|l@{\hspace{10pt}}|| *{5}{c|}| *{20}{c|}}
%\footnotesize
\hline
 & \multicolumn{5}{c||}{Queries} & \multicolumn{7}{c|}{Manipulation operations} \\
	& \rot{\samp} & \rot{\pro} & \rot{\eq}  & \rot{\inprod} & \multicolumn{1}{R{90}{0em}||}{\fid}
	& \rot{\addi} & \rot\had & \rot{\xyz} & \rot\cz & \rot{\swap} & \rot{\loc} & \rot{\T-gate} \\
\hline
Vector& \Yar & \Yes & \Yes & \Yes & \Yes & \Yes & \Yes & \Yes & \Yes & \Yes & \Yes & \Yes \\
\hline
%		| Sampl	| Prob 	| Eq	|Inprod	| Fid
ADD   	& \Yar	& \Yes	& \Yes	& \Yes & \Yes
%		| Add	| H		| XYZ	| CX	| Swap	| Local	| T
		& \Yes	& \Yes 	& \Yes	& \Yes	& \Yes	& \Yes	& \Yes \\
\hline
QMDD
		& \Yar	& \Yes	& \Yes	& \Yes & \Yes
		& \No	& \No 	& \Yes	& \Yes	& \No	& \No	& \Yes \\
\hline 
%QMDD  & \Yes & \Yes & \Yes & \No & \Yes & \No & ?? & ??  & ?? & \No & ?? & ??   \\
%\hline    
\limdd 	& \Yar	& \Yes	& \Yes	& \Cond & \Cond
		& \No	& \No	& \Yes	& \Yes	& \No	& \No	& \Yes  \\
\hline 
%TN    & \Cond? & \Cond? & \Cond? & \Cond? & ?? & \Yes? & \Yes? & \Yes? & \Yes? & \Yes? & \Yes? \\ \hline
MPS   & \Yar & \Yes & \Yes & \Yes & \Yes & \Yes 
	  & \Yes & \Yes & \Yes & \Yes & \Yes & \Yes  \\
\hline 
RBM   & \Yar    & ? & ? & \Cond & \Cond & ? & ? & \Yes & \Yes & \Yes & ? & \Yes \\
\hline 
%\multicolumn{13}{c}{Low priority} \\
%\hline 
%ZX    & ?? & ?? & ?? & ?? & ?? & ?? & ?? & ?? & ?? & ?? & ?? & ?? \\
%\hline 
%SLDD$_+$  & \Yes? & \Yes? & \Yes? & \Yes? & \Yes? & \No! & \No? & \Yes? & \No! & \No! & \No? & \Yes! \\
%\hline 
\end{tabular}
%\captionof{table}{
%From \cite{qkr2023}: Tractability of various queries and transformations on the data structures analyzed in this paper (single application of the operation). 
%\Yes means the data structure supports the operation in polytime,\\
%\Yar means supported in randomized polytime, and 
\footnotesize
\\
\hspace{-1.em}
\Yes(\No) means the operation is (not) supported in polytime.
\\
\phantom{ZZ}
\Cond means the operation is not supported unless $P=NP$.
%? is unknown.
%%\textbf{Fidelity} (not shown) has the same tractability as \textbf{Inprod} for all data structures considered.
%%\todo[inline]{still true?}
%The table only considers deterministic algorithms (for some ? a probabilistic algorithm exists, e.g., for \inprod on RBM).
%}
%\label{tab:tractability}
}
	\end{column}
\end{columns}
\vspace{1em}
\vspace{-1em}
\end{theorem}


\printbibliography[section=\therefsection]

\end{frame}
\end{refsection}




\begin{refsection}
\begin{frame}{Fidelity}
	\begin{theorem}
		RBM and \limdd are not tractable for \textbf{Fidelity}, unless ETH fails.
	\end{theorem}

\pause

\textbf{Proof:}
Let $\ket{D^k_n}$ be the Dicke state on $n$ qubits where basis states corresponding to  non-zero amplitudes have Hamming weight $k$, e.g. $\ket{D^2_3} \equiv \ket{110} + \ket{101} + \ket{011}$.


\pause

Let $G_n$ be an $n$-vertex graph and $\ket{G_n}$ be the corresponding graph state.

\pause

\alert{The \#EVEN SUBGRAPHS problem reduces to computing the fidelity $|\braket{D^k_n| G_n}|^2$}


\begin{definition}[\#EVEN SUBGRAPHS]
	\textbf{Input:} A graph $G=(V,E)$, an integer $k$\\
	\textbf{Output:} The number of induced $k$-vertex subgraphs with an even number of edges
\end{definition}


\pause

\begin{theorem}[\citeauthor{jerrum2017parameterised}]
	\label{thm:even-subgraphs-ETH-hard}
	If \#EVEN SUBGRAPHS is not in \P, unless ETH is false.
\end{theorem}

\onslide<+->{\alert{Actually, omitting detail: \#EVEN-ODD-SUBGRAPH-DIFFERENCE}}

\printbibliography[section=\therefsection]

\end{frame}
\end{refsection}





\begin{frame}{Circuit Simulation vs Applying Individual Gates}


\begin{alertblock}{Downsides of operation tractability}
	\begin{itemize}
	\item<+-> Only compares worst cases.
	\begin{itemize}
	\item ADD is tractable for all gates
	\item QMDD is intractable for e.g. Hadamard
	\item However, there is no state for which QMDD takes longer than ADD!
	\end{itemize}
	\item<+-> It tells us nothing about circuit simulation.
\begin{itemize}
	\item The circuit might double at every step.
%	\item 
\end{itemize}
	\end{itemize}
\end{alertblock}

\centering 
\setlength{\tabcolsep}{2pt}
\def\arraystretch{1.1}
\footnotesize
%\centering
\begin{tabular}{|l@{\hspace{10pt}}|| *{5}{c|}| *{20}{c|}}
%\footnotesize
\hline
 & \multicolumn{5}{c||}{Queries} & \multicolumn{7}{c|}{Manipulation operations} \\
	& \rot{\samp} & \rot{\pro} & \rot{\eq}  & \rot{\inprod} & \multicolumn{1}{R{90}{0em}||}{\fid}
	& \rot{\addi} & \rot\had & \rot{\xyz} & \rot\cz & \rot{\swap} & \rot{\loc} & \rot{\T-gate} \\
\hline
Vector& \Yar & \Yes & \Yes & \Yes & \Yes & \Yes & \Yes & \Yes & \Yes & \Yes & \Yes & \Yes \\
\hline
%		| Sampl	| Prob 	| Eq	|Inprod	| Fid
ADD   	& \Yar	& \Yes	& \Yes	& \Yes & \Yes
%		| Add	| H		| XYZ	| CX	| Swap	| Local	| T
		& \Yes	& \Yes 	& \Yes	& \Yes	& \Yes	& \Yes	& \Yes \\
\hline
QMDD
		& \Yar	& \Yes	& \Yes	& \Yes & \Yes
		& \No	& \No 	& \Yes	& \Yes	& \No	& \No	& \Yes \\
\hline 
%QMDD  & \Yes & \Yes & \Yes & \No & \Yes & \No & ?? & ??  & ?? & \No & ?? & ??   \\
%\hline    
\limdd 	& \Yar	& \Yes	& \Yes	& \alert{\Cond} & \alert{\Cond}
		& \No	& \No	& \Yes	& \Yes	& \No	& \No	& \Yes  \\
\hline 
%TN    & \Cond? & \Cond? & \Cond? & \Cond? & ?? & \Yes? & \Yes? & \Yes? & \Yes? & \Yes? & \Yes? \\ \hline
MPS   & \Yar & \Yes & \Yes & \Yes & \Yes & \Yes 
	  & \Yes & \Yes & \Yes & \Yes & \Yes & \Yes  \\
\hline 
RBM   & \Yar    & ? & ? & \alert{\Cond} & \alert{\Cond} & ? & ? & \Yes & \Yes & \Yes & ? & \Yes \\
\hline 
%\multicolumn{13}{c}{Low priority} \\
%\hline 
%ZX    & ?? & ?? & ?? & ?? & ?? & ?? & ?? & ?? & ?? & ?? & ?? & ?? \\
%\hline 
%SLDD$_+$  & \Yes? & \Yes? & \Yes? & \Yes? & \Yes? & \No! & \No? & \Yes? & \No! & \No! & \No? & \Yes! \\
%\hline 
\end{tabular}

\vspace{-.5em}
\footnotesize
\Yes means supported in (randomized \Yar) polytime\\
% means supported in randomized polytime, and 
\No means (not) supported in polytime
\\
\Cond means not supported in polytime conditionally

\end{frame}



\begin{frame}{Rapidity for Non-Canonical Data Structures}
\begin{refsection}

\vspace{-.5em}

\begin{itemize}
	\item Rapidity compares performance of operations on individual states
	\item Invented by Lai, Liu and Yin~\cite{lai2017new}
	\item We provide a more general definition
\end{itemize}


\begin{theorem}[\alert{omitting details}]
	If there is a polynomial time translation from data structure $D_1$ to $D_2$ and back, 
	
	and if there is a runtime monotonic algorithm implementing OP,
	
	then almost all operations on $D_1$ can be done as fast as on $D_2$.
\end{theorem}

\vspace{-.5em}
\centering

\scalebox{.75}{
 \begin{tikzpicture}[ele/.style={fill=black,circle,minimum width=.8pt,inner sep=1pt},every fit/.style={ellipse,draw,inner sep=-2pt}]


  \node[ele,label=left:$x_1$] (x1) at (0,2.5) {};    
	 \node[ele] (a1) at (0,1) {};

  \node[ele,,label=left:$x_2$] (x2) at (6.5,2.5) {};
  \node[ele,,label=right:$f(x_1)$] (fx1) at (4,2.5) {};
	 \node[ele,] (a2) at (6.5,1) {};
	 \node[ele,] (a2prime) at (4,1) {};

  \node[draw,gray!60,fit= (x1) (a1) ,minimum width=2cm,minimum height=3cm, label={above:$D_1$}] {};] {} ;
  \node[draw,gray!60,fit= (x2) (fx1) (a2prime) (a2),minimum width=5cm, minimum height=3cm,label={above:$D_2$}] {};]] {} ;

	 \draw[->,bend left=20,thick,shorten <=2pt,shorten >=2pt] (x1) to node[midway,fill=white] {$f$} (fx1);
  \draw[->,thick,shorten <=2pt,shorten >=2pt] (fx1) to node[midway,fill=white,solid] {$ALG_2^{rm}$} (a2prime);
  \draw[->,thick,shorten <=2pt,shorten >=2pt] (x2) to node[midway,fill=white,solid] {$ALG_2$} (a2);
	 \draw[->,bend left=20,thick,shorten <=2pt,shorten >=2pt] (a2prime) to node[midway,fill=white,solid] {$g$} (a1);

 \end{tikzpicture}
 }

\onslide<+->{
\centering

\renewcommand\qmdd{QMDD}


\centering
\scalebox{.7}{
	\begin{tikzpicture}[node distance=.5cm,minimum height=.5cm]

		\node[draw] (mps) 			   {MPS};
		\node[draw, left = 1cm of mps] (limdd) {\limdd};
		\node[draw,  below = of limdd] (qmdd) {\qmdd};
		\node[draw, below = of qmdd] (add) {ADD};

		\draw[\bigarrowhead, solid] (qmdd.north)     to (limdd.south);
		\draw[\bigarrowhead, solid] (add.north)      to (qmdd.south);
		\draw[\bigarrowhead, solid, bend right = 20] (qmdd.east) to (mps.south);
	\end{tikzpicture}
}
%	\caption{Rapidity relations between data structures considered here.
%	A solid arrow $D_1\to D_2$ means $D_2$ is at least as rapid as $D_1$ 
%	for all operations satisfying \ref{i:omega}~and~\ref{i:rm}~of~\cref{thm:sufficient-condition-rapidity}.
%	}
%	\label{fig:rapidity}


%\begin{theorem}
%	MPS is never much slower than QMDD for circuit simulation.
%\end{theorem}
%
%\vspace{-.5em}
%
%\begin{theorem}
%	\limdd is never much slower than QMDD for circuit simulation.
%\end{theorem}
}

\vspace{-.5em}

\printbibliography[section=\therefsection]

\end{refsection}
\end{frame}




%\begin{refsection}
%\begin{frame}{Applications of Decision Diagrams / Satisfiability}
%	
%	\begin{block}{Application Areas}
%\begin{itemize}
%	\item Simulation~\cite{vinkhuijzen2021limdd,limdd2}
%	\item Equivalence Checking of (Clifford) Circuits~\cite{thanos2023fast}
%	\item (Graph State) Circuit Synthesis~\cite{brand2023quantum}
%%	\item 
%\end{itemize}
%\end{block}	
%\printbibliography[section=\therefsection]
%\end{frame}
%\end{refsection}


%\begin{frame}{Take Aways}
%
%\begin{block}{Decision Diagrams in Quantum Computing}
%\begin{itemize}[<+->]
%	\item Better data structures yield better circuit simulation, equivalence checks \& synthesis
%	\item \limdd unites the strengths of decision diagrams and the stabilizer formalism
%	\item \limdd, MPS and RBM are incomparable in terms of succinctness
%	\item Rapidity tells us that MPS and \limdd are never much slower than QMDD, ADD
%\end{itemize}
%\end{block}
%
%\centering
%
%\begin{tikzpicture}\footnotesize
%  \tikzset{venn circle/.style={circle,minimum width=2cm,fill=####1,opacity=0.4}}
%  \node [venn circle=white,minimum width=4cm,draw] (A) at (0,0.3) {};
%  \node  at (0,1.95) 			{State space};
%
%  \node [venn circle = Red!40!white, ellipse,minimum height=2.2cm, minimum width=3.6cm] (L) at (0,0.6) {};
%  \node  at (0,1.5) 		{\limdd};
%  
%  \node [venn circle = blue!70!white,text width=1.3cm,align=center,rotate=79,ellipse,minimum height=1.8cm, minimum width=2.5cm] (B) at (-.6,-.2) {};
%  \node[text width=1.3cm,align=center]  at (-.9,-1.) {MPS};
%
%
%  \node [venn circle = green!70!white,text width=1.3cm,align=center,rotate=119,ellipse,minimum height=1.8cm, minimum width=2.5cm] (B) at (.6,-.2) {};
%  \node[text width=1.3cm,align=center]  at (.9,-1.) {RBM};
%
%
%
%  \node [venn circle = Blue!100!white,text width=1cm,align=center, minimum width=1.cm] (B) at (-.5,.3) 	{\textcolor{white}{QMDD}};						
%
%  \node [venn circle = OliveGreen!100!white,text width=1cm,align=center, minimum width=1.cm] (C) at (.5,.2) {\textcolor{white}{Stabilizer states}};
%
%\end{tikzpicture}
%~~~~~
%\includegraphics[width=3cm]{limdd}
%
%\end{frame}






