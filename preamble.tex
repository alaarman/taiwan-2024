
%\RequirePackage{color}
%\RequirePackage[utf8]{inputenc}
%\RequirePackage{graphicx}
%\RequirePackage{epigraph}
%\RequirePackage{amsmath}

%\RequirePackage[ruled,vlined]{algorithm2e}
\RequirePackage{algorithm}
\RequirePackage[noend]{algpseudocode}

\RequirePackage{amssymb}
\RequirePackage{braket}
%\RequirePackage{xfrac} %for sfrac
\usepackage{eqparbox}

%\let\proof\relax
%\let\endproof\relax
%\RequirePackage{amsthm}
%\RequirePackage{thmtools}
\RequirePackage{multirow}
%\RequirePackage{comment}
%\RequirePackage[T1]{fontenc}
\RequirePackage{stmaryrd}
%\RequirePackage{thm-restate}
%\RequirePackage{complexity}
%\RequirePackage[colorlinks,citecolor=blue!60!black!70,linkcolor=blue!60!black!70]{hyperref}%\RequirePackage{cite}
%\RequirePackage{hyperref}
%%\RequirePackage{caption}
%%\RequirePackage{subcaption}
%\RequirePackage{mathrsfs}
%\RequirePackage[english]{babel}
%%\RequirePackage{mathabx}
%%\RequirePackage{pgfplots}
\RequirePackage{calc}
\RequirePackage{nicefrac}
\RequirePackage{complexity}
%\RequirePackage{fdsymbol}
%\RequirePackage{eucal}
\RequirePackage{tikz}
\usetikzlibrary{shapes,backgrounds,calc,arrows,automata}
\usetikzlibrary{graphs,quotes}
\usetikzlibrary{arrows.meta}
\usetikzlibrary{positioning}
\usetikzlibrary{decorations.pathreplacing}
%\usetikzlibrary{shapes,backgrounds,calc,arrows}
%\RequirePackage{verbatim}
\RequirePackage{soul}
%\RequirePackage{wrapfig}
%\tikzstyle{every node} = [circle, draw,inner sep=3pt]
\let\clipbox\relax
\usepackage{adjustbox}
\usepackage{array}
\usepackage{booktabs}
\usepackage{multicol} 
\usepackage{empheq}
\usepackage{xspace}
\usepackage{textpos}
\usepackage{pifont} 
\usepackage{eurosym} 

\RequirePackage{venndiagram}

\usepackage{comment}
\usepackage{qcircuit}
\usepackage{circuitikz}
\usepackage{tabto}



\newenvironment<>{varblock}[2][\textwidth]{
    \begin{center}
      \begin{minipage}{#1}
        \setlength{\textwidth}{#1}
          \begin{actionenv}#3
            \def\insertblocktitle{#2}
            \par
%            \usebeamertemplate{block begin}
            }
  {\par
%      \usebeamertemplate{block end}
    \end{actionenv}
  \end{minipage}
\end{center}}

%\usepackage{enumitem}


%\newlist{todolist}{itemize}{2}
%\setlist[todolist]{label=$\square$}
%\usepackage{pifont}


\newcommand{\cmark}{\ding{51}}%
\newcommand{\xmark}{\ding{55}}%
\newcommand{\done}{\rlap{$\square$}{\raisebox{2pt}{\large\hspace{1pt}\cmark}}%
\hspace{-2.5pt}}
\newcommand{\wontfix}{\rlap{$\square$}{\large\hspace{1pt}\xmark}}


\usepackage{etoolbox}

\newcommand{\algruledefaultfactor}{.75}
\newcommand{\algstrut}[1][\algruledefaultfactor]{\vrule width 0pt
depth .25\baselineskip height #1\baselineskip\relax}
\newcommand*{\algrule}[1][\algorithmicindent]{\hspace*{.5em}\vrule\algstrut
\hspace*{\dimexpr#1-.5em}}

\makeatletter
\newcount\ALG@printindent@tempcnta
\def\ALG@printindent{%
    \ifnum \theALG@nested>0% is there anything to print
    \ifx\ALG@text\ALG@x@notext% is this an end group without any text?
    % do nothing
    \else
    \unskip
    % draw a rule for each indent level
    \ALG@printindent@tempcnta=1
    \loop
    \algrule[\csname ALG@ind@\the\ALG@printindent@tempcnta\endcsname]%
    \advance \ALG@printindent@tempcnta 1
    \ifnum \ALG@printindent@tempcnta<\numexpr\theALG@nested+1\relax% can't do <=, so add one to RHS and use < instead
    \repeat
    \fi
    \fi
}%

\patchcmd{\ALG@doentity}{\noindent\hskip\ALG@tlm}{\ALG@printindent}{}{\errmessage{failed to patch}}

\AtBeginEnvironment{algorithmic}{\lineskip0pt}

\newcommand\Stateh{\State \algstrut[1]}



\interfootnotelinepenalty=10000

\RequirePackage[bottom,symbol]{footmisc} 
\RequirePackage{footnote}
\makesavenoteenv{tabular}
\makesavenoteenv{table}

%\let\OldComment\Comment
%\renewcommand\Comment[1]{\OldComment {\footnotesize #1}}
\renewcommand{\algorithmiccomment}[1]{\hfill$\triangleright$ {\footnotesize #1}}


\newlength\mytemplen
\newsavebox\mytempbox

\makeatletter
\newcommand\mybluebox{%
    \@ifnextchar[%]
       {\@mybluebox}%
       {\@mybluebox[0pt]}}

\def\@mybluebox[#1]{%
    \@ifnextchar[%]
       {\@@mybluebox[#1]}%
       {\@@mybluebox[#1][0pt]}}

\def\@@mybluebox[#1][#2]#3{
    \sbox\mytempbox{#3}%
    \mytemplen\ht\mytempbox
    \advance\mytemplen #1\relax
    \ht\mytempbox\mytemplen
    \mytemplen\dp\mytempbox
    \advance\mytemplen #2\relax
    \dp\mytempbox\mytemplen
    \colorbox{myblue}{\hspace{1em}\usebox{\mytempbox}\hspace{1em}}}
\makeatother


\definecolor{shadecolor}{cmyk}{.05,.05,0.05,0.05}
\definecolor{light-blue}{cmyk}{0.15,.15,.15,.15}
\newsavebox{\mysaveboxM} % M for math
\newsavebox{\mysaveboxT} % T for text

\newcommand*\Garybox[2][Example]{%
  \sbox{\mysaveboxM}{#2}%
  \sbox{\mysaveboxT}{\fcolorbox{black}{light-blue}{#1}}%
  \sbox{\mysaveboxM}{%
    \parbox[b][\ht\mysaveboxM+.5\ht\mysaveboxT+.5\dp\mysaveboxT][b]{\wd\mysaveboxM}{#2}%
  }%
  \sbox{\mysaveboxM}{%
    \fcolorbox{black}{shadecolor}{%
      \makebox[19em]{\usebox{\mysaveboxM}}%
    }%
  }%
  \usebox{\mysaveboxM}%
  \makebox[0pt][r]{%
    \makebox[\wd\mysaveboxM][c]{%
      \raisebox{\ht\mysaveboxM-0.5\ht\mysaveboxT+0.5\dp\mysaveboxT-0.5\fboxrule}{\usebox{\mysaveboxT}}%
    }%
  }%
}


\makeatletter
\def\blfootnote{\gdef\@thefnmark{}\@footnotetext}
\makeatother


%\RequirePackage{textgreek} % for \textchi

%\usepackage{algorithm}
%\usepackage{algpseudocode}



%\theoremstyle{definition}
%\newtheorem{definition}{Definition}
%\declaretheorem[name=Theorem]{theorem}
%\newtheorem{lemma}[theorem]{Lemma}
%\newtheorem{corollary}{Corollary}[theorem]
%\newtheorem{proposition}{Proposition}
%\newtheorem{proposition}{Proposition}
%\newtheorem{conjecture}{Conjecture}
%\declaretheorem[name=Conjecture]{conjecture}
%\declaretheorem{definition}
%\newtheorem{problem}{Problem}
%\newtheorem{subproblem}{Problem}[problem]
%\newtheorem{solution}{Solution}[problem]
%\newtheorem{subsolution}{Solution}[subproblem]
%\newtheorem{conjecturesolution}{Solution}[conjecture]
%\newtheorem{conjecturecounterexample}{Counterexample}[conjecture]
%\newtheorem{smalldefinition}{Definition}

%\declaretheoremstyle[qed=$\diamond$]{definitionstyle}
%\theoremstyle{definitionstyle}
%\declaretheorem[style=definitionstyle]{definitions}
%


%%%% New commands for algorithmic package: Case statements
\algnewcommand\algorithmicswitch{\textbf{switch}}
\algnewcommand\algorithmiccase{\textbf{case}}
\algnewcommand\algorithmicassert{\texttt{assert}}
\algnewcommand\Assert[1]{\State \algorithmicassert(#1)}%
% New "environments"
\algdef{SE}[SWITCH]{Switch}{EndSwitch}[1]{\algorithmicswitch\ #1\ \algorithmicdo}{\algorithmicend\ \algorithmicswitch}%
\algdef{SE}[CASE]{Case}{EndCase}[1]{\algorithmiccase\ #1}{\algorithmicend\ \algorithmiccase}%
\algtext*{EndSwitch}%
\algtext*{EndCase}%

%




\setlength{\parskip}{7pt}
\setlength{\parindent}{0pt}


% Below: added by Tim

\usepackage{enumerate}
\newcommand{\innerprod}[2]{\langle #1 | #2 \rangle}

\newcommand{\unit}{1\!\!1}
\def\X{X}
\def\Y{Y}
\def\Z{Z}
\newenvironment{algo}[1][Algorithm]
    {\begin{center} 
    \begin{tabular}{|p{0.9\textwidth}|} 
    \hline 
    \vspace{0.1\baselineskip} 
    \textbf{#1}\\} 
    { 
    \\\hline 
    \end{tabular} 
    \end{center} 
    \vspace{\baselineskip} 
    } 

%\theoremstyle{df}
%\declaretheorem{df}
%\theoremstyle{thm}
%\declaretheorem{thm}
%\newtheorem{example}{Example}
%\theoremstyle{prop}
%\declaretheorem{prop}
%\theoremstyle{corollary}
%\declaretheorem{corollary}
%\theoremstyle{conj}
%\declaretheorem{conj}

%\usepackage{physics}
% package physics clashes with another package; now manually adding the
% commands needed:
%\newcommand{\expval}[1]{\langle #1 \rangle}
%\newcommand{\dyad}[1]{| #1 \rangle \langle #1 |}
%\newenvironment{smallmat}{\left[\begin{smallmatrix}}{\end{smallmatrix}\right]}
%\newcommand{\Iso}{\ensuremath{\text{Iso}}}
%\newcommand{\Stab}{\ensuremath{\text{Stab}}}
%\newcommand{\Aut}{\ensuremath{\text{Aut}}}

%\defaccr{\qmdd}{\textsf{QMDD}}
%\defaccr{\add}{\textsf{ADD}}
%\defaccr{\isoqmdd}{\textsf{LIMDD}}
%\defaccr{\limdd}{\textsf{LIMDD}}
%\defaccr{\qmdds}{\textsf{QMDD}s}
%\defaccr{\adds}{\textsf{ADD}s}
%\defaccr{\isoqmdds}{\textsf{LIMDD}s}
%\defaccr{\limdds}{\textsf{LIMDD}s}
%\defaccr{\glimdd}{\ensuremath{G}-\limdd}
%\defaccr{\glimdds}{\ensuremath{G}-\limdds}

\newcommand{\diagonal}[1]{\begin{smallmat}1 & 0 \\ 0 & #1\end{smallmat}}
\newcommand{\antidiagonal}[1]{\begin{smallmat}0 & #1 \\ 1 & 0\end{smallmat}}

%\newcommand{\tim}[1]{\textcolor{blue}{[\textbf{Tim: }#1]}}
%\newcommand{\lieuwe}[1]{\textcolor{blue}{\textbf{Lieuwe: }#1}}

%\renewcommand\index{\textsf{idx}}
%\newcommand\iso{\textsf{iso}}




% begin vertical rule patch for algorithmicx (http://tex.stackexchange.com/questions/144840/vertical-loop-block-lines-in-algorithmicx-with-noend-option)
\makeatletter
% start with some helper code
% This is the vertical rule that is inserted
\renewcommand*{\algrule}[1][\algorithmicindent]{\makebox[#1][l]{\hspace*{.5em}\vrule height .75\baselineskip depth .25\baselineskip}}%
\makeatother



%\newcommand[1]{\genset}{[#1]}
\usepackage{caption}
